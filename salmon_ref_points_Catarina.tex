


\documentclass[10pt]{article}

% Allow Unicode input (alternatively, you can use XeLaTeX or LuaLaTeX)
\usepackage[utf8]{inputenc}
\usepackage{natbib} 
\usepackage{setspace}
\usepackage[parfill]{parskip} % no indent

\usepackage[margin=0.8in]{geometry}



\newcommand{\ts}{\textsuperscript}
\onehalfspacing
\title{Paragraph for Robyn with references}
\author{}

\date{\today}

\begin{document}
\maketitle



The above calculations apply to iteroparous species, where we need to account for  mortality of mature fish as they age and spawn through the years. We do not have to account for this for semelparous species, such as some salmonids (e.g., Pacific salmon), which spawn only once in their lifetime. For these species, the population can be modeled using only the stock-recruit relationship,  often the Ricker \citep{rickerStockRecruitment1954} model, under the assumption that the estimate of recruits are equivalent to the spawners in the subsequent generation, in the absence of fishing. MSY-based reference points can be estimated directly from the alpha and beta parameters of the Ricker or Beverton-Holt stock-recruit models (see \citet[Table 7.2]{hilbornQuantitativeFisheriesStock1992}, \citet{scheuerellExplicitSolutionCalculating2016} for $S_{MSY}$ and $U_{MSY}$, and \citet{holtIndicatorsStatusBenchmarks2009,holtEvaluationBenchmarksConservation2009} for $S_{gen}$ ). In Canada, reference points for Pacific salmon populations are frequently based on assessments which rely on various metrics to assign status. See \citet{pestalStateSalmonRapid2022}, previous Wild Salmon Policy assessments \citep[e.g.,][]{grant2017FraserSockeye2020,dfoIntegratedBiologicalStatus2016,dfoWildSalmonPolicy2015}, and the recent papers on Limit Reference Points (LRPs) for Pacific salmon major stocks \citep{holtGuidelinesDefiningLimitInpress,holtCaseStudyApplicationsInpress,dfoMethodologiesGuidelinesDefining2022}. For Atlantic salmon, an iteroparous species, reference points are typically based on the egg deposition that results in less than an acceptably low probability that the realized smolt production from freshwater would be less than 50\% of the estimated maximum recruitment \citep{dfoDevelopmentReferencePoints2015,dfoDefinitionPrecautionaryApproach2022}. We do not have time to explore estimation of salmonid reference points in these webinars but, in addition to the previous references, see \citet{chaputConsiderationsDefiningReference2012,portleyLimitReferencePoints2014,grantEvaluationUncertaintyFraser2011,grantIntegratedBiologicalStatus2013,holtImpactTimevaryingProductivity2020} for more information. *RF will ask colleagues for some key refs*


\bibliographystyle{apa} 
\bibliography{RP_101_salmon.bib}


\end{document}